\section{Research proposal}\label{sec:proposal}

\subsection{Description of the proposed research (weight 40\%)}\label{sec:research}

\subsubsection{Overall aim and key objective}\label{sec:aim}

\paragraph{Introduction}\label{sec:intro}

Text \cite{clarke2018}.

\begin{mdframed}
    \mdheader{Frame header}
    Some text.
\end{mdframed}

\subsubsection{Research plan}\label{sec:plan}


\subsubsection{Motivation for choice of institute}\label{sec:institute}


\vfill
\vspace{\baselineskip}
(\ref{sec:aim}, \ref{sec:plan}, and \ref{sec:institute} combined total: max. 2,000 words on max. six pages)\\

Total word count \ref{sec:aim}, \ref{sec:plan}, and \ref{sec:institute}: \\

%---------------------------------------------------------

\newpage
\subsection{Knowledge utilisation (weight 20\%)}\label{sec:KU}
\checkedbox[0pt] Yes, this proposal has the potential of knowledge utilization\\
\checkbox[0pt] No, this proposal has no direct knowledge utilization\\

%\textbf{If yes}, please describe the potential of knowledge utilization, including: \vspace{-6pt}
%\begin{itemize}
%    \item[--] Contribution to society and/or other scientific areas;
%    \item[--] Disciplines and organisations that might benefit from the results.
%\end{itemize}

%Additionally, describe the implementation, including: \vspace{-6pt}
%\begin{itemize}
%    \item[--] Action plan to allow the outcomes of the research project to benefit the potential knowledge users;
%    \item[--] If and how the potential knowledge users will be involved;
%    \item[--] (concrete) outcomes for society and/or other academic disciplines;
%    \item[--] The period over which knowledge utilisation is expected to occur.
%\end{itemize}

%\textbf{If no}, please motivate why your proposal has no direct knowledge utilization.\\
%\vspace{\baselineskip}

\vfill
\vspace{\baselineskip}
(\ref{sec:KU}: max. 700 words on max. 2 pages)\\

Word count \ref{sec:KU}:

%---------------------------------------------------------

\newpage
\subsection{Literature references}\label{sec:ref}

\newrefcontext[sorting=nyt]
\printbibliography[heading=none,notcategory=fullcited]

%---------------------------------------------------------

\newpage
\subsection{Data management}\label{sec:dataman}

\begin{enumerate}
    \item Will this project involve re-using existing research data?
    \item[] \checkedbox[1em] Yes: Are there any constraints on its re-use?
    \item[] \checkbox[1em] No: Have you considered re-using existing data but discarded the possibility? Why?
    \vspace{6pt}
    
%    \item[] If no, please briefly explain why; if yes, state any constraints on re-use of existing data if there are any.
%    \vspace{6pt}
    
    \item[] Text.
    \vspace{6pt}
    
    \item Will data be collected or generated that are suitable for reuse?
    \item[] \checkedbox[1em] Yes: Please answer questions 3 and 4.
    \item[] \checkbox[1em] No: Please explain why the research will not result in reusable data or in data that cannot be stored or
    \item[] \hspace*{3.4em}data that for other reasons are not relevant for reuse. 
    \vspace{6pt}
    
    \item After the project has been completed, how will the data be stored for the long-term and made available for the use by third parties? Are there possible restrictions to data sharing or embargo reasons? Please state these here.%
    \vspace{6pt}
    
    \item[] Text.
    \vspace{6pt}
    
    \item Will any costs (financial and time) related to data management and sharing/preservation be incurred?
    \item[] \checkbox[1em] Yes: Then please be sure to specify the associated expenses in the budget table of this proposal.
    \item[] \checkedbox[1em] No: All the necessary resources (financial and time) to store and prepare data for sharing/preservation
    \item[] \hspace*{3.4em}are or will be available at no extra cost.
\end{enumerate}


