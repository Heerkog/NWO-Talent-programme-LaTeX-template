\section{Research proposal}\label{sec:proposal}

\subsection{Description of the proposed research}\label{sec:research}

\subsubsection{Overall aim and key objective}\label{sec:aim}

\paragraph{Introduction}\label{sec:intro}

Text \cite{clarke2018}.

\begin{mdframed}
    \mdheader{Frame header}
    Some text.
\end{mdframed}

\subsubsection{Research plan}\label{sec:plan}


\subsubsection{Motivation for choice of institute}\label{sec:institute}


%---------------------------------------------------------

\newpage
\subsection{Scientific and/or societal impact of the proposed project}\label{sec:KU}
% Describe the proposed impact of the research within the stated maximum number of words  (max. 1,000 words, please see the Explanatory Notes for further explanations).

% Choose one of these:
Primary focus on scientific impact
% Scientific and societal impact are of comparable focus
% Primary focus on societal impact

% Please elaborate on the scientific and/or societal impact of the proposed project:
% When writing, please consider the following points:
%     • Motivate your choice for the focus on scientific and/or societal impact;
%     • Elaborate on and motivate all applicable parts of the criteria for the chosen form of impact (see call for proposals);
%     • If applicable: indicate how attention will be paid to the non-chosen form of impact (thus, even if the focus is on scientific impact, the applicant should still describe how they will pay attention to possible or unforeseen societal impact of the research, and vice versa).

\subsection{Number of words}\label{sec:proposal}
% Indicate the number of words used. Words in references, footnotes, figures, figure captions and tables should be included in the count.

Section 2a: Wordcount:     (max. 4,000)\\
Section 2b: Wordcount:     (max. 1,000)

%---------------------------------------------------------

\newpage
\subsection{Literature references}\label{sec:ref}

\newrefcontext[sorting=nyt]
\printbibliography[heading=none,notcategory=fullcited]

%---------------------------------------------------------

\newpage
\subsection{Data management}\label{sec:dataman}

\begin{enumerate}
    \item Will data be collected or generated that are suitable for reuse?
    \item[] \checkedbox[1em] Yes: Then answer questions 2 to 4.
    \item[] \checkbox[1em] No: Then explain why the research will not result in reusable data or in data that cannot be stored or data that for other reasons are not relevant for reuse.
    \vspace{6pt}
    \item Where will the data be stored during the research?
    \vspace{6pt}
    \item[] Text.
    \vspace{6pt}

    \item After the project has been completed, how will the data be stored for the long-term and made available for the use by third parties? To whom will the data be accessible?
    \vspace{6pt}
    \item[] Text.
    \vspace{6pt}

    \item Which facilities (ICT, (secure) archive, refrigerators or legal expertise) do you expect will be needed for the storage of data during the research and after the research? Are these available?
    \vspace{6pt}
    \item[] Text.
    \vspace{6pt}
\end{enumerate}


%%% Local Variables: 
%%% coding: utf-8
%%% mode: latex
%%% TeX-engine: xetex
%%% TeX-master: "NWO_template"
%%% End: 